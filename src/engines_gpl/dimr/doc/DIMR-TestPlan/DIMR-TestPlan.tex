\documentclass[signature]{deltares_manual}
\svnid{$Id: DIMR-TestPlan.tex 48961 2016-12-09 10:52:28Z peele_rh $}
\usepackage{booktabs}
\usepackage {color}
\definecolor {gray}     {rgb} { 0.4, 0.4, 0.4 }
\definecolor {darkblue} {rgb} { 0.0, 0.0, 0.6 }
\definecolor {cyan}     {rgb} { 0.0, 0.6, 0.6 }
\definecolor {orange}   {rgb} { 1.0, 0.5, 0.0 }
\definecolor {brown}    {rgb} { 0.6, 0.3, 0}
\usepackage {listings}

\lstdefinelanguage {XML} {
	escapechar=\%,
	identifierstyle=\color{brown},
	stringstyle=\color{blue},
	morestring=[b]",
	morecomment=[s]{<?}{?>},
	morecomment=[s][\color{green}]{<!--}{-->},
	keywordstyle=\color{red},
	morekeywords={
		designation,
		icaoCode,
		ilsFreq,
		ilsID,
		name,
		partition,
		pathname,
		range,
		schemaLocation,
		state,
		tora,
		towerFreq,
		trueBRG,
		type,
		units,
		value,
		version,
		xmlns,
		xsi
	} % list your attributes here
}

\lstset {
	language=XML,
	basicstyle=\ttfamily\footnotesize,
	columns=fullflexible,
	showstringspaces=false,
	commentstyle=\color{gray}\upshape
}

%------------------------------------------------------------------------------
\graphicspath{{../common/}}
\usepackage[final]{pdfpages}
%------------------------------------------------------------------------------
\hypersetup
{
	pdfauthor   = {Deltares},
	pdftitle    = {DIMR},
	pdfkeywords = {Deltares Test Plan DIMR},
	pdfcreator  = {LaTeX hyperref},
	pdfproducer = {ps2pdf}
}
%------------------------------------------------------------------------------

\renewcommand\BackgroundPicPart{
	\put(0,0){
		\parbox[b][\paperheight]{\paperwidth}{%
			\vspace{8\baselineskip}
			\hspace{44mm} % 20mm wegens leftmarginskip + 24 mm ~ 1 inch
			\includegraphics[width=\paperwidth, height=\paperheight, keepaspectratio]{../common/pictures/deltares_delft3d_part.png}%
			\vfill
		}
	}
}
\renewcommand\BackgroundPicChapter{
	\put(0,0){
		\parbox[b][\paperheight]{\paperwidth}{%
			\vspace{4\baselineskip}
			\hspace{220mm}
			\includegraphics[width=15mm]{../common/cover/d-hydro_ribbon_page_manual.pdf}%
			\vfill
		}
	}
}

\newcommand\T{\rule{0pt}{2.6ex}}       % Top T
\newcommand\B{\rule[-1.2ex]{0pt}{0pt}} % Bottom T

\usepackage[titletoc]{appendix}
% D-Waves
\newcommand{\dwaves}{D-Waves\xspace}
\newcommand{\swan}{SWAN\xspace}
% D-WAQ
\newcommand{\dwaqfull}{\textrm{D-Water~Quality}\xspace}
\newcommand{\dwaq}{\dwaqfull}
% Flow modules
\newcommand{\dflowfmfull}{\textrm{D-Flow~Flexible~Mesh}\xspace}
\newcommand{\dflowfm}{\textrm{D-Flow~FM}\xspace}
\newcommand{\dflowoned}{D-Flow\,1D\xspace}
% D-RR
\newcommand{\drainfallrunoff}{\textrm{D-Rainfall~Runoff}\xspace}
\newcommand{\drrfull}{\drainfallrunoff}
\newcommand{\drr}{D-RR\xspace}
% D-RTC
\newcommand{\drtcfull}{D-Real~Time~Control\xspace}
\newcommand{\drtc}{D-RTC\xspace}
\newcommand{\rtctools}{\textrm{RTC-Tools}\xspace}
\newcommand{\fbctools}{\textrm{FBC-Tools}\xspace}



\newcommand{\dimr}{\textrm{DIMR}\xspace}
\newcommand{\dimrfull}{\textrm{Deltares Integrated Model Runner}\xspace}
\newcommand{\ProgramName}{\dimrfull (\dimr)\xspace}

%-----------------------------------------------
%\lstset{ %
%	basicstyle=\footnotesize,        % the size of the fonts that are used for the code listings
%}

\begin{document}
\pagestyle{empty}
\includepdf[pages=1,offset=72 -70]{../common/cover/deltaressystems-cover-dimr-um.pdf} % links-rechts past precies
\cleardoublepage
%
\title{\ProgramName}
\subtitle{Test Plan}
\manualtype{}
\projectnumber{DC DSC - Testbank Activiteiten, DIMR improvements, ts 182}
\client{Deltares}
\reference{I1000511-009}
\classification{-}

\date{2020-10-30}
\version{1.0}

\keywords{software \dimrfull \dimr}

\references{Refer to \autoref{chap:literature}}
	
\summary{\dimr is a small program to run combinations of simulation cores, components, ensuring that data is exchanged at the right timing.}

\versioni{1.0}
\datei{2020-10-30}
\authori{Adri Mourits,\newline Ralph Peelen}
\revieweri{Luca Carniato,\newline Stef Hummel,\newline Arjen Markus}
\approvali{Arthur Baart,\newline Jan Mooiman}
\status{Concept}
\deltarestitle
%------------------------------------------------------------------------------
\chapter{Introduction} 
\label{chapterIntroduction}

\section{Purpose and scope of this document} \label{sec:PurposeAndScope}

\section{Related Documents}
\label{sec:RelatedDocuments}
\bigskip
\begin{longtable}{|p{0.28\textwidth}|p{0.72\textwidth}|}
\caption{Overview of related documents for the \ProgramName \label{tab:RelatedDocuments}}\\	\hline
		\hline 
		\textbf{Document} & \textbf{Reference} \\
		\hline 
		\hline 
		Functional Design & \citep{PROJECT_NAME_FunctionalDesignPROJECT_YEAR} \\
		Technical Design & \citep{PROJECT_NAME_TechnicalDesignPROJECT_YEAR} \\
		Technical Documentation & \citep{PROJECT_NAME_TechnicalDocumentationPROJECT_YEAR} \\
		Test Report & \citep{PROJECT_NAME_TestReportPROJECT_YEAR}  \\
    \hline			
\end{longtable}

\section{Versions}
\label{sec:Versions}

\subsection{Version 0.1}
\label{sec:Version}
This is the draft version of this document

\section{Requirements}
\label{sec:requirements}
To do.


\section{Test levels}
\label{sec:testLevels}
Based on the V-model approach we will be testing the following levels:
\begin{enumerate}
\item Component testing (\autoref{unitTest}).
\item Integration testing (\autoref{integrationTest}).
\item System testing (\autoref{systemTest}).
\item Acceptance level (\autoref{acceptanceTest}).
\end{enumerate}

\section{Code coverage}
\label{sec:codeCoverage}
To-do
%------------------------------------------------------------------------------
\chapter{Component Testing (Unit tests)}
\label{unitTest}
To-do

%------------------------------------------------------------------------------
\chapter{Integration Testing (Integration tests)}
\label{integrationTest}
To-do

%------------------------------------------------------------------------------
\chapter{System Testing}
\label{systemTest}
To-do

%------------------------------------------------------------------------------
\chapter{Acceptation level (Acceptance tests)}
\label{acceptanceTest}
To-do

%------------------------------------------------------------------------------
\chapter{Literature}  \label{chapterLiterature}

\bibliography{../PROJECT_NAME_references/PROJECT_NAME_references}

\pagestyle{empty}
\mbox{}

%------------------------------------------------------------------------------
\end{document}
